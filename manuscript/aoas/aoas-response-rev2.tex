\documentclass[12pt]{article}

\usepackage{amsmath,amssymb,amstext}
\usepackage[margin=1in]{geometry}
\usepackage[round,sort&compress]{natbib}
\usepackage[letterpaper=true,colorlinks=true,pdfpagemode=none,urlcolor=red,linkcolor=red,citecolor=red,pdfstartview=FitH]{hyperref}

\usepackage{xcolor}
\definecolor{cobalt}{rgb}{0.0, 0.28, 0.67}

\newenvironment{resp}{\color{cobalt}}{}
\newcommand{\sresp}[1]{\textcolor{cobalt}{#1}}

\usepackage{xr}
\externaldocument{musicManuscript-rev1}
\externaldocument[supp-]{../music-supplement-rev1}


\title{AOAS2002-013, \emph{Markov-switching state space models for uncovering
    musical interpretation} \\
Author response - Revision 2}
\author{Daniel J McDonald, Michael McBride, Yupeng Gu, Christopher Raphael}
\date{\today}


\begin{document}


\maketitle


We would again like to thank the editor and the reviewer for their careful
comments. In this minor revision, we have implemented the main suggestions
highlighted by the editor. The major change is an extra section in the
Supplemental Material addressing the question of ``over smoothing'' (second
bullet point below). We've made a handful of other minor adjustments discussed
below. 

\begin{itemize}
\item ``Bayesian analysis'' - We added the following 2 sentences to Section 2.5
  where we first describe their inclusion in our analysis:
  \begin{quote}
    Given the use of priors, one could interpret our procedure from a Bayesian
    perspective. However, we do not estimate full posteriors, only the maximum
    \emph{a posteriori} parameter value, so our prior specification choices are
    not intended to reflect tail uncertainty.
  \end{quote}
  We would prefer to avoid characterizing our procedure as ``Bayesian''. We
  feel that adding such language sets up expectations in the reader that
  accounting for posterior uncertainty will form a component of the analysis. We
  would rather not give this impression.

\item We went ahead and fit the model of Gu and Raphael (2012). We give a few
  results in the Supplement with additional discussion. Here are the highlights:
  \begin{enumerate}
  \item We show three performances and the fits from both models
    side-by-side. Our model produces more reasonable inferences and also
    smooths less. The GR 2012 model tends to get ``stuck'' in a state,
    especially in the Luisada recording, eliminating much of the tempo
    variability.
  \item We examine the cumulative RMSE for both models for the same
    3 performances. The GR 2012 model has larger RMSE in each case, approaching
    the RMSE of the mean for the Luisada recording. It's fairly clear that it
    smooths significantly more than our more expressive model.
  \item We summarize these comparisons over all 46 recordings (lower RMSE on
    35/46 recordings, smaller average RMSE overall).
  \end{enumerate}

  
\item ``Mahalanobis distance'' Using the prosterior rather than the prior
  doesn't change the results 
  much, but they're not quite as good for the reasons previously discussed.
  Perhaps the reviewer thinks we're using the covariance rather than the
  precision? More likely, we suspect the reviewer's intuition is related to the
  question of 
  sampling from the 
  posterior rather than maximizing. Here, (maximizing) parameters that are
  poorly identified 
  remain near their initial values. And since the same initial values are used
  for all performances, poorly identified parameters have very low variance in
  the posterior over the set of performances. So the inversion inflates their
  importance. Had we sampled, one 
  would imagine that poor identification would lead to a posterior that looks
  like the prior, so it would no longer be an issue. We haven't altered our
  analysis.

\item We've taken the advice of the editor and added the color version to the
  supplement along with a mention in the caption in the manuscript.

\item As the editor remarks, we do conduct a sensitivity analysis in Section
  3.6. There's very little 
  difference in the results between our use of gamma and the more standard
  inverse gamma. When 
  calibrated to match means and variances (as we do), the bulk of the distribution is
  nearly indistinguishable. The gamma prior allows for smaller variances while
  the inverse gamma 
  puts a bit more mass at very large values. Typical values of the objective
  function (negative log likelihood plus + log prior) are on the order of 1000
  while the absolute differences in log gamma relative to log inverse gamma over
  the effective range of the variance parameters are on the order of 0.1.
  
\end{itemize}


\end{document}

