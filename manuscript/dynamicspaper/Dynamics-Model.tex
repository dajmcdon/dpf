% !TeX program = pdfLaTeX
\documentclass[12pt]{article}
\usepackage{amsmath}
\usepackage{graphicx,psfrag,epsf}
\usepackage{enumerate}
\usepackage{natbib}
\usepackage{textcomp}
\usepackage[hyphens]{url} % not crucial - just used below for the URL
\usepackage{hyperref}
\providecommand{\tightlist}{%
  \setlength{\itemsep}{0pt}\setlength{\parskip}{0pt}}

%\pdfminorversion=4
% NOTE: To produce blinded version, replace "0" with "1" below.
\newcommand{\blind}{0}

% DON'T change margins - should be 1 inch all around.
\addtolength{\oddsidemargin}{-.5in}%
\addtolength{\evensidemargin}{-.5in}%
\addtolength{\textwidth}{1in}%
\addtolength{\textheight}{1.3in}%
\addtolength{\topmargin}{-.8in}%

%% load any required packages here




\usepackage{amsmath}
\usepackage{booktabs}
\usepackage{pgf}
\usepackage{tikz}

\begin{document}


\def\spacingset#1{\renewcommand{\baselinestretch}%
{#1}\small\normalsize} \spacingset{1}


%%%%%%%%%%%%%%%%%%%%%%%%%%%%%%%%%%%%%%%%%%%%%%%%%%%%%%%%%%%%%%%%%%%%%%%%%%%%%%

\if0\blind
{
  \title{\bf Using Switching State-Space Models to Interpret Musical Dynamics}

  \author{
        Robert Granger \thanks{The authors gratefully acknowledge \ldots{}} \\
    Department of Statistics, Indiana University\\
      }
  \maketitle
} \fi

\if1\blind
{
  \bigskip
  \bigskip
  \bigskip
  \begin{center}
    {\LARGE\bf Using Switching State-Space Models to Interpret Musical Dynamics}
  \end{center}
  \medskip
} \fi

\bigskip
\begin{abstract}
The text of your abstract. 200 or fewer words.
\end{abstract}

\noindent%
{\it Keywords:} 3 to 6 keywords, that do not appear in the title
\vfill

\newpage
\spacingset{1.45} % DON'T change the spacing!

\hypertarget{introduction}{%
\section{Introduction}\label{introduction}}

Things to write about in this section

\begin{enumerate}
\def\labelenumi{\arabic{enumi})}
\item
  Classifying music performances to sort on personal preference
\item
  Smoothing performances to sound better from Midi data
\item
  Using State-Switching Model
\end{enumerate}

With a piece of classical music, directions such as

\section{The Switching State-Space Model}
\label{sec:model}

State-Space models are commonly used to model time series observations
in the presence of perceived hidden, continuous states. As a result of
the state-space framework, the observations are viewed as independent
conditional on the hidden states whereas these hidden states will follow
a vector autoregressive process. Adding assumptions of linearity and
Gaussian error produces the following model commonly referred to as the
general linear Gaussian state-space model \citet{durbin_time_2012} takes
the form \begin{equation}
  \begin{aligned}
    y_t &= C_t + D_tx_t + \epsilon_t, 
    & \epsilon_t & \sim N(0,G_t)\\
    x_{t+1} &= A_t + B_tx_t + \eta_t, 
    & \eta_t & \sim N(0,H_t), 
    & x_1 & \sim N(x_0,P_0) \\
  \end{aligned}
  \label{eq:statespacemod}
\end{equation} where the first part of \autoref{eq:statespacemod} is
known as the observation equation and the second part is known as the
state equation. \(y_t\) is a vector of known observations, and \(x_t\)
is a vector of the unobserved, continuous states at each time period,
\(t\). The observation error, \(\epsilon_t\), and the state equation
error, \(\eta_t\), are both assumed to be independent and identically
distributed.

In the typical state-space framework, the matrices \(A_t\), \(B_t\),
\(C_t\), \(D_t\), \(G_t\), and \(H_t\) are allowed to vary across time
but are known. If there are a finite number of perceived structures for
these matrices, and the given structure is unknown at time, \(t\), a
switching state-space model can be used. This model assumes there are
some underlying discrete states, \(s_i\), that transition over time
through a Markov process. Making this slight adjustment to
\autoref{eq:statespacemod} yields the following model which will be used
as a basic framework for modeling music dynamics.

\begin{equation}
  \begin{aligned}
    y_t &= C_t(\Theta_t) + D_t(\Theta_t)x_t + \epsilon_t, 
    & \epsilon_t & \sim N(0,G_t)\\
    x_{t+1} &= A_t(\Theta_t) + B_t(\Theta)x_t + \eta_t, 
    & \eta_t & \sim N(0,H_t), 
    & x_1 & \sim N(x_0,P_0) \\
  \end{aligned}
  \label{eq:switchstatemodel}
\end{equation}

When it comes to musical dynamics, rarely do musicians attempt to play
with the same dynamics or loudness throughout the entirety of a piece.
While there may be dramatic changes occasionally throughout the
performance, most of the time we expect the musician to steadily change
the loudness from note to note. Furthermore, the musician may quietly or
loudly play individual notes for emphasis. In order to model this
behavior, we propose the following four discrete states:

\begin{list}{}{}

\item[$s_1$:] The musician selects a new value for loudness.

\item[$s_2$:] The musician continues the dynamics in a steady way.

\item[$s_3$:] The musician plays a single note more loudly.

\item[$s_4$:] The musician plays a single note more softly.

\end{list}

The observation, \(y_t\) is the univariate loudness of the note at each
time period, \(t\). In order to allow the dynamics to progress steadily,
we implement the framework of \citet{gu_modeling_2012} where the
continuous hidden states follow a process that allows for piece-wise
quadratic displays with \[x_t = (x^0_t, x^1_t, x^2_t), \] where
\(x^0_t\), \(x^1_t\), and \(x^2_t\) are the loudness, the first order
difference, and the second order difference, respectively. Since we want
to maintain this smooth progression even when the musician plays a
single note more loudly or softly, states \(s_3\) and \(s_4\) are
implemented by adding a constant in the observation equation as opposed
to changing the state equation. \autoref{tab:parmats} shows the
parameter matrices for the four states.

The final part of the switching state-space model is designing the
Markov process for transitioning between the discrete states.
\autoref{fig:transmat} displays the structure of the transition between
states. The first state, \(s_1\), allows for the selection of a new
loudness which then transitions into the smooth progresson state,
\(s_2\), with probability 1. When arriving in \(s_2\), the next time
period's discrete state can be any of the possible states including
itself. If in this progression, we move to a sudden loud note, \(s_3\),
or a sudden soft note, \(s_4\), then we have the opportunity to continue
the smooth progression, \(s_2\), or start a new smooth progression,
\(s_1\). It is not permissible though to immediately then play another
sudden loud or soft note. With four states, the transition matrix could
potentially have 12 probabilities to estimate; however, because of
restrictions placed on the possible transitions, only 5 probabilities
need to be estimated.

\begin{table}
\centering
\begin{tabular}[h!]{@{}llccccccc@{}}
\toprule
%&&&\multicolumn{3}{c}{Parameter Matrices}\\
  \multicolumn{2}{c}{States} &\phantom{a}& \multicolumn{6}{c}{Parameter Matrices}\\
  \cmidrule{1-2} \cmidrule{4-9}
  $s_i$ &&& $A$ & $B$ & $C$ & $D$ & $G$ & $H$ \\
  \midrule
  $s_1$ & && $\begin{pmatrix} \mu_1 \\ \mu_2 \\ \mu_3 \end{pmatrix}$ & $\begin{pmatrix} 0&0&0 \\ 0&0&0 \\ 0&0&0 \end{pmatrix}$ & 0 & $\begin{pmatrix} 1\\ 0 \\ 0 \end{pmatrix}$ & $\sigma_\epsilon^2$ & $\begin{pmatrix} \sigma_1^2&0&0 \\ 0&\sigma_2^2&0 \\ 0&0&\sigma_3^2 \end{pmatrix}$\\
  \\
  $s_2$ & && $\begin{pmatrix} 0 \\ 0 \\ 0 \end{pmatrix}$ & $\begin{pmatrix} 1&1&0 \\ 0&1&1 \\ 0&0&1 \end{pmatrix}$ & 0 & $\begin{pmatrix} 1\\ 0 \\ 0 \end{pmatrix}$ & $\sigma_\epsilon^2$ & $\begin{pmatrix} 0&0&0 \\ 0&0&0 \\ 0&0&0 \end{pmatrix}$\\
  \\
  $s_3$ & && $\begin{pmatrix} 0 \\ 0 \\ 0 \end{pmatrix}$ & $\begin{pmatrix} 1&1&0 \\ 0&1&1 \\ 0&0&1 \end{pmatrix}$ & $\mu_c$ & $\begin{pmatrix} 1\\ 0 \\ 0 \end{pmatrix}$ & $\sigma_\epsilon^2$ & $\begin{pmatrix} 0&0&0 \\ 0&0&0 \\ 0&0&0 \end{pmatrix}$\\
  \\
  $s_4$ & && $\begin{pmatrix} 0 \\ 0 \\ 0 \end{pmatrix}$ & $\begin{pmatrix} 1&1&0 \\ 0&1&1 \\ 0&0&1 \end{pmatrix}$ & $-\mu_c$ & $\begin{pmatrix} 1\\ 0 \\ 0 \end{pmatrix}$ & $\sigma_\epsilon^2$ & $\begin{pmatrix} 0&0&0 \\ 0&0&0 \\ 0&0&0 \end{pmatrix}$\\

\bottomrule
\end{tabular}
\caption{Parameter matrices for the switching state space model.\label{tab:parmats}}
\end{table}

\begin{figure}[tb!]
  \centering
  \tikzstyle{switch}=[rectangle,
  thick, minimum size=1cm, draw=black]
  \begin{tikzpicture}[>=latex,text height=1.5ex,text depth=0.25ex]
    \matrix[row sep=0.25cm,column sep=.5cm] {
      &&& \node (S1) [switch] {$s_1$};&&& \\
      \\ \\ \\ \\ \\ \\
      &&& \node (S2) [switch] {$s_2$};&&& \\
      \\ \\ \\ \\ \\ \\
      &\node (S3) [switch] {$s_3$}; &&&& \node (S4) [switch] {$s_4$};\\
    };
    \path[->]
    (S1) edge [bend right] node [left] {1}(S2)
    (S2) edge [bend right] node [right] {$p_{21}$}(S1)
    (S2) edge [loop above] node [left] {}(S2)
    (S2) edge [bend right] node [right] {$p_{23}$}(S3)
    (S2) edge [bend right] node [right] {$p_{24}$}(S4)
    (S3) edge [bend left] node [left] {$p_{31}$}(S1)
    (S3) edge [bend right] node [left] {}(S2)
    (S4) edge [bend right] node [right] {$p_{41}$}(S1)
    (S4) edge [bend right] node [left] {}(S2);
  \end{tikzpicture}
  \caption{Transition diagram. \label{fig:transmat}}
\end{figure}

\section{Evaluating the Model}
\label{sec:analysis}

Things to include:

\begin{enumerate}
\def\labelenumi{\arabic{enumi})}
\tightlist
\item
  Discussion of state-space model algorithm with discrete particle
  filter
\end{enumerate}

\section{Conclusion}
\label{sec:conclusion}

\nocite{*}

\bibliographystyle{agsm}
\bibliography{musicdynamicsbib.bib}

\end{document}
