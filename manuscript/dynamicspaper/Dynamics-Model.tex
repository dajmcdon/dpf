% !TeX program = pdfLaTeX
\documentclass[12pt]{article}
\usepackage{amsmath}
\usepackage{graphicx,psfrag,epsf}
\usepackage{enumerate}
\usepackage{natbib}
\usepackage{textcomp}
\usepackage[hyphens]{url} % not crucial - just used below for the URL
\usepackage{hyperref}
\providecommand{\tightlist}{%
  \setlength{\itemsep}{0pt}\setlength{\parskip}{0pt}}

%\pdfminorversion=4
% NOTE: To produce blinded version, replace "0" with "1" below.
\newcommand{\blind}{0}

% DON'T change margins - should be 1 inch all around.
\addtolength{\oddsidemargin}{-.5in}%
\addtolength{\evensidemargin}{-.5in}%
\addtolength{\textwidth}{1in}%
\addtolength{\textheight}{1.3in}%
\addtolength{\topmargin}{-.8in}%

%% load any required packages here




\usepackage{amsmath}
\usepackage{booktabs}
\usepackage{pgf}
\usepackage{tikz}
\usepackage{algorithm2e}

\begin{document}


\def\spacingset#1{\renewcommand{\baselinestretch}%
{#1}\small\normalsize} \spacingset{1}


%%%%%%%%%%%%%%%%%%%%%%%%%%%%%%%%%%%%%%%%%%%%%%%%%%%%%%%%%%%%%%%%%%%%%%%%%%%%%%

\if0\blind
{
  \title{\bf Using Switching State-Space Models to Interpret Musical Dynamics}

  \author{
        Robert Granger \thanks{The authors gratefully acknowledge \ldots{}} \\
    Department of Statistics, Indiana University\\
      }
  \maketitle
} \fi

\if1\blind
{
  \bigskip
  \bigskip
  \bigskip
  \begin{center}
    {\LARGE\bf Using Switching State-Space Models to Interpret Musical Dynamics}
  \end{center}
  \medskip
} \fi

\bigskip
\begin{abstract}
The text of your abstract. 200 or fewer words.
\end{abstract}

\noindent%
{\it Keywords:} 3 to 6 keywords, that do not appear in the title
\vfill

\newpage
\spacingset{1.45} % DON'T change the spacing!

\hypertarget{introduction}{%
\section{Introduction}\label{introduction}}

\def\algorithmautorefname{Algorithm}

Things to write about in this section

\begin{enumerate}
\def\labelenumi{\arabic{enumi})}
\item
  Classifying music performances to sort on personal preference
\item
  Smoothing performances to sound better from Midi data
\item
  Using State-Switching Model
\end{enumerate}

The composer gives directions about dynamics. \(p\) stands for ``piano''
and indicates this section should be played ``soft''. The ``f'' stands
for ``forte'' and indicates this section should be played

\section{The Switching State-Space Model}
\label{sec:model}

State-Space models are commonly used to model time series observations
in the presence of perceived hidden, continuous states. As a result of
the state-space framework, the observations are viewed as independent
conditional on the hidden states whereas these hidden states will follow
a vector autoregressive process. Adding assumptions of linearity and
Gaussian error produces the following model commonly referred to as the
general linear Gaussian state-space model \citet{durbin_time_2012} takes
the form \begin{equation}
  \begin{aligned}
    y_t &= C_t + D_tx_t + \epsilon_t, 
    & \epsilon_t & \sim N(0,G_t)\\
    x_{t+1} &= A_t + B_tx_t + \eta_t, 
    & \eta_t & \sim N(0,H_t), 
    & x_1 & \sim N(x_0,P_0) \\
  \end{aligned}
  \label{eq:statespacemod}
\end{equation} where the first part of \autoref{eq:statespacemod} is
known as the observation equation and the second part is known as the
state equation. \(y_t\) is a vector of known observations, and \(x_t\)
is a vector of the unobserved, continuous states at each time period,
\(t\). The observation error, \(\epsilon_t\), and the state equation
error, \(\eta_t\), are both assumed to be independent and identically
distributed.

In the typical state-space framework, the matrices \(A_t\), \(B_t\),
\(C_t\), \(D_t\), \(G_t\), and \(H_t\) are allowed to vary across time
but are known. If there are a finite number of perceived structures for
these matrices, and the given structure is unknown at time, \(t\), a
switching state-space model can be used. This model assumes there are
some underlying discrete states, \(s_i\), that transition over time
through a Markov process. Making this slight adjustment to
\autoref{eq:statespacemod} yields the following model which will be used
as a basic framework for modeling music dynamics.

\begin{equation}
  \begin{aligned}
    y_t &= C_t(s_t) + D_t(s_t)x_t + \epsilon_t, 
    & \epsilon_t & \sim N(0,G_t)\\
    x_{t+1} &= A_t(s_t) + B_t(s_t)x_t + \eta_t, 
    & \eta_t & \sim N(0,H_t), 
    & x_1 & \sim N(x_0,P_0) \\
  \end{aligned}
  \label{eq:switchstatemodel}
\end{equation}

When it comes to musical dynamics, rarely do musicians attempt to play
with the same dynamics or loudness throughout the entirety of a piece.
While there may be dramatic changes occasionally throughout the
performance, most of the time we expect the musician to steadily change
the loudness from note to note. Furthermore, the musician may quietly or
loudly play individual notes for emphasis. In order to model this
behavior, we propose the following four discrete states:

\begin{list}{}{}

\item[$s^1$:] The musician selects a new value for loudness.

\item[$s^2$:] The musician continues the dynamics in a steady way.

\item[$s^3$:] The musician plays a single note more loudly.

\item[$s^4$:] The musician plays a single note more softly.

\end{list}

The observation, \(y_t\) is the univariate loudness of the note at each
time period, \(t\). In order to allow the dynamics to progress steadily,
the continuous hidden states, \(x_t\), follow a process that allows for
piece-wise quadratic displays. At each time period, \(x_t\), is an array
of length three, \[x_t = (x^0_t, x^1_t, x^2_t), \] where \(x^0_t\) is
the loudness, \(x^1_t\) is the first order difference, and \(x^2_t\) is
the second order difference. Since we want to maintain this smooth
progression even when the musician plays a single note more loudly or
softly, states \(s^3\) and \(s^4\) are implemented by adding a constant
in the observation equation as opposed to changing the state equation.
The model is similar to that proposed in \citet{gu_modeling_2012}, but
extended to include these additional states. \autoref{tab:parmats} shows
the parameter matrices for the four states.

The final part of the switching state-space model is designing the
Markov process for transitioning between the discrete states.
\autoref{fig:transmat} displays the structure of the transition between
states. The first state, \(s^1\), allows for the selection of a new
loudness which then transitions into the smooth progresson state,
\(s^2\), with probability 1. When arriving in \(s^2\), the next time
period's discrete state can be any of the possible states including
itself. If in this progression, we move to a sudden loud note, \(s^3\),
or a sudden soft note, \(s^4\), then we have the opportunity to continue
the smooth progression, \(s^2\), or start a new smooth progression,
\(s^1\). It is not permissible though to immediately then play another
sudden loud or soft note. With four states, the transition matrix could
potentially have 12 probabilities to estimate; however, because of
restrictions placed on the possible transitions, only 5 probabilities
need to be estimated.

In total, there are 13 distinct unknown parameters,
\(\theta \in \Theta\), spread across 4 unknown states, \(s^i \in S\),
through \(T\) time periods. A summary of the parameters to be estimated
are:
\(\Theta = \{\mu_c, \sigma^2_\epsilon, \mu_0, \mu_1, \mu_2, \sigma^2_0, \sigma^2_1, \sigma^2_2, p_{21}, p_{23}, p_{24}, p_{31}, p_{41}\}\).
In the next section, we discuss finding smoothed estimates of the
dynamics which requires estimation of these parameters along with the
continous and discrete states.

\begin{table}
\centering
\begin{tabular}[h!]{@{}llccccccc@{}}
\toprule
%&&&\multicolumn{3}{c}{Parameter Matrices}\\
  \multicolumn{2}{c}{States} &\phantom{a}& \multicolumn{6}{c}{Parameter Matrices}\\
  \cmidrule{1-2} \cmidrule{4-9}
  $S$ &&& $A$ & $B$ & $C$ & $D$ & $G$ & $H$ \\
  \midrule
  $s^1$ & && $\begin{pmatrix} \mu_0 \\ \mu_1 \\ \mu_2 \end{pmatrix}$ & $\begin{pmatrix} 0&0&0 \\ 0&0&0 \\ 0&0&0 \end{pmatrix}$ & 0 & $\begin{pmatrix} 1\\ 0 \\ 0 \end{pmatrix}$ & $\sigma_\epsilon^2$ & $\begin{pmatrix} \sigma_0^2&0&0 \\ 0&\sigma_1^2&0 \\ 0&0&\sigma_2^2 \end{pmatrix}$\\
  \\
  $s^2$ & && $\begin{pmatrix} 0 \\ 0 \\ 0 \end{pmatrix}$ & $\begin{pmatrix} 1&1&0 \\ 0&1&1 \\ 0&0&1 \end{pmatrix}$ & 0 & $\begin{pmatrix} 1\\ 0 \\ 0 \end{pmatrix}$ & $\sigma_\epsilon^2$ & $\begin{pmatrix} 0&0&0 \\ 0&0&0 \\ 0&0&0 \end{pmatrix}$\\
  \\
  $s^3$ & && $\begin{pmatrix} 0 \\ 0 \\ 0 \end{pmatrix}$ & $\begin{pmatrix} 1&1&0 \\ 0&1&1 \\ 0&0&1 \end{pmatrix}$ & $\mu_c$ & $\begin{pmatrix} 1\\ 0 \\ 0 \end{pmatrix}$ & $\sigma_\epsilon^2$ & $\begin{pmatrix} 0&0&0 \\ 0&0&0 \\ 0&0&0 \end{pmatrix}$\\
  \\
  $s^4$ & && $\begin{pmatrix} 0 \\ 0 \\ 0 \end{pmatrix}$ & $\begin{pmatrix} 1&1&0 \\ 0&1&1 \\ 0&0&1 \end{pmatrix}$ & $-\mu_c$ & $\begin{pmatrix} 1\\ 0 \\ 0 \end{pmatrix}$ & $\sigma_\epsilon^2$ & $\begin{pmatrix} 0&0&0 \\ 0&0&0 \\ 0&0&0 \end{pmatrix}$\\

\bottomrule
\end{tabular}
\caption{Parameter matrices for the switching state space model.\label{tab:parmats}}
\end{table}

\begin{figure}[tb!]
  \centering
  \tikzstyle{switch}=[rectangle,
  thick, minimum size=1cm, draw=black]
  \begin{tikzpicture}[>=latex,text height=1.5ex,text depth=0.25ex]
    \matrix[row sep=0.25cm,column sep=.5cm] {
      &&& \node (S1) [switch] {$s_1$};&&& \\
      \\ \\ \\ \\ \\ \\
      &&& \node (S2) [switch] {$s_2$};&&& \\
      \\ \\ \\ \\ \\ \\
      &\node (S3) [switch] {$s_3$}; &&&& \node (S4) [switch] {$s_4$};\\
    };
    \path[->]
    (S1) edge [bend right] node [left] {1}(S2)
    (S2) edge [bend right] node [right] {$p_{21}$}(S1)
    (S2) edge [loop above] node [left] {}(S2)
    (S2) edge [bend right] node [right] {$p_{23}$}(S3)
    (S2) edge [bend right] node [right] {$p_{24}$}(S4)
    (S3) edge [bend left] node [left] {$p_{31}$}(S1)
    (S3) edge [bend right] node [left] {}(S2)
    (S4) edge [bend right] node [right] {$p_{41}$}(S1)
    (S4) edge [bend right] node [left] {}(S2);
  \end{tikzpicture}
  \caption{Transition diagram. \label{fig:transmat}}
\end{figure}

\section{Evaluating the Model}
\label{sec:analysis}

\subsection{The algorithm}

The construction of the model is designed to allow us to separate the
observed dynamics, \(y_t\), from the performer's intended dynamics,
\(E[y_t]\). Hence, the aim in estimating the intended dynamics is
attempting to remove \(\epsilon_t\), which can be seen as unintended
deviations from the desired loudness.

When the parameter values, \(\theta_i \in \Theta\), and the discrete
states, \(S\), are known, the Kalman filter (see
\citet{kalman_new_1960}) can be used to find estimates of the continuous
hidden states, \(\{x_t\}_{t=0}^t\), along with the likelihood of
\(\Theta\). While the Kalman filter provides an easy way to compute the
likelihood, the estimate of \(x_t\) is obtained using only observations
coming before time \(t\). In order to use all information, we implement
the Kalman smoothing algorithm introduced in \citet{rauch_maximum_1965}.
Given \(\Theta\) and \(S\), this algorithm provides an estimate,
\(\hat{x_t} = E[x_t|y_0,...y_T]\), which can be used to compute a fitted
or ``smoothed'' value, \(\hat{y_t}\).

So far, if \(\Theta\) and \(S\) are known, we have a closed form
solution to obtaining the smoothed dynamics through the Kalman smoother.
If \(\Theta\) is known, we can obtain the most likely set of discrete
states, \(S\), by simply running the Kalman smoother algorithm on every
possible combination of discrete states and discovering the largest
likelihood. This may work if the number of discrete states and/or time
periods is small; however, the number of state combinations to check
would be \(\|\{s_i \in S\}\|^T\). For example, if we were using the
music dynamics model presented in the previous section on Chopin's
Mazurka Op. 63 No.~3 (which has 231 notes), the total number of state
combinations to check would be \(4^{231}\approx1.19\times 10^{139}\). Of
course, many of these state combinations could be removed due to 0
likelihood given the restrictions on the state transitions, but even if
this is taken into account, the number of state combinations would be
far too large. To overcome this issue, we only look at a subset of these
combinations through the Discrete Particle Filter as described in
\citet{mcdonald_markov-switching_2019}. The algorithm works by
estimating the partial likelihood iteratively through time at each of
the possible discrete state paths using the Kalman Filter. Each of these
paths is known as a particle with only the previous states and partial
likelihood being saved. This process would still require checking all
\(\{\#s_i \in S\}^T\) paths, so to get around this, a maximum number of
particles to save at each time iteration is selected.

\begin{figure}[tb!]
  \centering
  \tikzstyle{switch}=[rectangle,
  thick, minimum size=0.5cm, draw=black]
  \begin{tikzpicture}[>=latex,text height=1.5ex,text depth=0.25ex]
    \matrix[row sep=0.25cm,column sep=.75cm] {
      \node (S1) [switch] {$s_1$}; &&\node (S4) [switch] {$s_1s_1$};&& \node (S7) [switch] {$s_1s_1$}; && \node (S8) [switch] {$s_1s_1s_1$}; && \node (S9) [switch] {$s_1s_1s_1$}; &&\\
      \\
      && && && \node (S13) [switch] {$s_1s_1s_2$};\\
      \\
       &&\node (S3) [switch] {$s_1s_2$}; && \node (S10) [switch] {$s_1s_2$}; && \node (S14) [switch] {$s_1s_2s_1$}; && \node (S20) [switch] {$s_1s_2s_1$}; \\
      \\ 
             && &&  && \node (S15) [switch] {$s_1s_2s_2$}; && \node (S21) [switch] {$s_2s_2s_2$}; \\
      \\
      \node (S2) [switch] {$s_2$}; &&\node (S5) [switch] {$s_2s_1$};&&\node (S11) [switch] {$s_2s_1$};&& \node (S16) [switch] {$s_2s_1s_1$}; && \node (S22) [switch] {$s_2s_1s_1$}; \\
      \\
      && && && \node (S17) [switch] {$s_2s_1s_2$};&&\node (S23) [switch] {$s_2s_1s_2$};
      \\ 
       &&\node (S6) [switch] {$s_2s_2$};&&\node (S12) [switch] {$s_2s_2$};&& \node (S18) [switch] {$s_2s_2s_1$};\\
      \\
      &&&&&& \node (S19) [switch] {$s_2s_2s_2$}; &&\\
      \\
      \hline
      \\
      \\
      $t=1$ && && \hspace{12px} $t=2$ && && \hspace{52px} $t=3$  \\
    };
    \path[->]
    (S1) edge (S3)
    (S7) edge (S8)
    (S7) edge (S13)
    (S8) edge [dashed] (S9)
    (S10) edge (S14)
    (S10) edge (S15)
    (S11) edge (S16)
    (S11) edge (S17)
    (S4) edge [dashed] (S7)
    (S1) edge (S4)
    (S2) edge (S5)
    (S12) edge (S18)
    (S12) edge (S19)
    (S3) edge [dashed] (S10)
    (S5) edge [dashed] (S11)
    (S6) edge [dashed] (S12)
    (S14) edge [dashed] (S20)
    (S15) edge [dashed] (S21)
    (S16) edge [dashed] (S22)
    (S17) edge [dashed] (S23)
    (S2) edge (S6);
  \end{tikzpicture}
  \caption{Discrete Particle Filter with 2 discrete states and capping the maximum number of stored particles at 5 for each time iteration. \label{fig:dpf}}
\end{figure}

\autoref{fig:dpf} illustrates this concept with 2 discrete states and
the maximum number of particles set at 5. There are four possible state
paths to check when passing from \(t=1\) to \(t=2\) which is less than
the maximum particle number so all four paths are saved. When moving
from \(t=2\) to \(t=3\), the number of paths increases to 8. Since only
5 paths are allowed, three of these paths must be dropped and will no
longer be considered as possible solutions. In order to decide which
paths are to be saved a sampling procedure must be performed. One
sampling procedure proposed by \citet{tugnait_detection_1982} is to
simply keep the paths that have the highest likelihood. Another sampling
procedure proposed by \citet{akashi_random_1977} is to randomly sample
one particle out of the total number of states descended from each of
the \(t-1\) particles in proportion to their respective likelihoods.
\citet{fearnhead_-line_2003} proposes another stochastic approach but
one that minimizes the expected mean squared error. This approach
determines a threshold value such that all particles with likelihood
above this value are kept and the remaining particles to be kept are
chosen at random with probability equal to their respective likelihoods.

The final step is estimating the model parameters, \(\Theta\), that
maximize the likelihood. This is no easy task as this model presents a
couple of problems: 1) many of the parameters are constrained in our
model as variances need to be positive and the probabilities should be
between 0 and 1; 2) there are many local maxima making finding a global
maxima difficult. To help alleviate these concerns, we can implement
Bayesian priors on our parameters. Using carefully selected priors
allows us to steer the algorithm away from impossible solutions and
direct it toward more desired solutions. These issues also lead us to
carefully consider our optimization technique. We will use two different
methods. First, we will implement Nelder-Mead which can find local
maxima for unconstrained multidimensional problems without the need for
derivative computation \citet{nelder_simplex_1965}. Second, we will
implement Simulated Annealing (SANN) which is commonly used when dealing
with problems with many local maxima in order to estimate the global
maxima. For both of these methods, the selection of initial values can
greatly determine the outcome, so many should be selected and tried. The
best solution found, regardless of the algorithm or initial values,
shall be deemed the approximate solution.

The complete algorithm can be found in \autoref{alg:masteralgorithm}.
The solution to this model was solved using software R
\citet{r_core_team_r_2019}. The Nelder-Mead optimization is performed
using the package \texttt{optimr} by \citet{nash_optimr_2019} and the
discrete particle filter was implemented by extending the package
\texttt{dpf} by \citet{mcdonald_dpf_2020}.

\begin{algorithm}[H]
 \caption{Algorithm for Solving Music Dynamics Model\label{alg:masteralgorithm}}
\SetAlgoLined
 \textbf{Input} Y, a vector of observed dynamics\;
 \textbf{Initialize} $\Theta$\;
 \While{(Stopping Criteria)}{
 Create Matrices A, B, C, D, G, H at each $t$ for each $s^i \in S$\;
 Implement Discrete Particle Filter\;
 Compute the log-likelihood: \hspace{2px} $l(Y|\Theta,S) + l(S|\Theta) + l(\Theta)$\;
 Update $\Theta$ via Nelder-Mead Optimization or SANN\;
 }
\end{algorithm}

\subsection{Results for Chopin's Mazurka Op. 63 No. 3}

The model was fit using observed dynamics from 46 different performances
of Chopin's Mazurka Op. 63. No 3. This data was created by Andrew Earis
\citet{earis_mazurka_2009} and found on the website for the Centre for
the History and Analysis of Recorded Music (CHARM)
\citet{charm_centre_2009}. The technique used to create the data is
described in detail in his paper ``An algorithm to extra expressive
timing and dynamics from piano recordings''
\citet{earis_algorithm_2007}.

To implement \autoref{alg:masteralgorithm} on dynamics data from
performances of Chopin's Mazurka Op. 63 No.3, we must make a few
specific decisons. First, we choose prior distributions applicable
specifically to Chopin's Mazurka Op. 63 No.3, which can be found in
\autoref{tab:priors}. Secondly, regarding the Discrete Partile Filter,
we selected to keep up to 500 particles with the sampling criterion of
keeping the largest likeliehoods. Lastily, since there are many local
minima, we used 20 different starting points drawn randomly and
independently from each of the prior distributions per optimization
technique (both Nelder-Mead and SANN).

\begin{table}[t]
  \centering
  \begin{tabular}{@{}rcll@{}}
    \toprule
    Parameter & \phantom{a} & Distribution & Prior Mean \\
    \midrule
    $\mu_c$ & $\sim$ & Gamma$(100,\ 0.1)$ & 10\\
    $\sigma^2_{\epsilon}$ & $\sim$ & Gamma$(10,\ 0.5)$ & 5\\
    $\mu_{0}$ & $\sim$ & Normal$(\overline{Y},\ 10)$ & $\overline{Y}$\\
    $\mu_{1} $ & $\sim$ & Normal$(0,\ 0.25)$ & 0\\
    $\mu_{2} $ & $\sim$ & Normal$(0,\ 0.25)$ & 0\\
    $\sigma^2_{0} $ & $\sim$ & Gamma$(10,\ 1)$ & 10 \\
    $\sigma^2_{1} $ & $\sim$ & Gamma$(3,\ 1)$ & 3 \\
    $\sigma^2_{2} $ & $\sim$ & Gamma$(3,\ 1)$ & 3 \\
    $p_{2,\cdot}$ & $\sim$ & Dirichlet$(5,\ 85,\ 5,\ 5)$ & 0.05, 0.85, 0.05, 0.05 \\
    $p_{3,\cdot}$ & $\sim$ & Beta$(2,\ 8)$ & 0.20\\
    $p_{4,\cdot}$ & $\sim$ & Beta$(2,\ 8)$ & 0.20\\
    \bottomrule
  \end{tabular}
  \caption{Informative prior distributions for Chopin's Mazurka Op. 63 No.3}
  \label{tab:priors}
\end{table}

The results for one performance by Sviatislav Richter are shown in
Figure \ref{fig:richter1976} with the results of the other performances
found in the appendix. The black line traces the observed dynamics for
each note across time in accordance with their note onset. The colored
dots are used to differentiate the 4 possible discrete states and show
the interpreted musical dynamics for each note. The background is shaded
based on musical directions given by the composer in the score. Notice
with this performance, the model often seems to start a new smooth
progression when the composer gives direction about the dynamics. There
are occasions where this is not always the case. Notice that Richter
appears to smoothly adjust the dynamics from the first ``forte'' section
into the first ``piano'' section. Then he proceeds with a couple of
abrupt changes to the dynamics although keeping it relavitely ``soft''.
These unscripted parts of the performance are what we hope to discover
as they can provide an explanation for why one person might prefer one
performance over another.

\begin{figure}
\centering
\includegraphics{Dynamics-Model_files/figure-latex/richter1976-1.pdf}
\caption{\label{fig:richter1976}Interpreted Dynamics from Richter's 1976
performance of Chopin's Mazurka Op. 63 No.~3. The black line traces the
observed dynamics while the colored dots are the smoothed interpreted
dynamics.}
\end{figure}

The final important piece of this model that can be used to characterize
a performance is the parameter values, \(\Theta\). The parameters for
this particular performance can be found in
\autoref{tab:parameterestimatesrichter}, and all parameter estimates for
the 46 performances can be found in the appendix. Certain parameters can
be very informative in differentiating performances. For example, large
values of \(\mu_c\) will tell us that the performer often uses very loud
or very soft points of emphasis, whereas smaller values indicate the
performer is more subtle. If we want to get some idea as to the overall
loudness of a piece, we may take a look at \(\mu_0\). The parameter
values, along with the interpreted dynamics and discrete state, provide
considerable information regarding differentiating between performances.

\begin{table}[t]
  \centering
  \begin{tabular}{@{}rcccccccc@{}}
    \toprule
    $\mu_c$ & $\sigma^2_{\epsilon}$ & $\mu_{0}$ & $\mu_{1}$ & $\mu_{2}$ & $\sigma^2_{0} $ & $\sigma^2_{1} $ & $\sigma^2_{2}$ \\
    \midrule
    $10.7833$ & $4.2609$ & $21.8404$ & $-0.1353$ & $-0.0222$ & $12.2454$ & $0.2225$ & $0.0034$ \\
    \midrule
    $p_{21}$ & $p_{23}$ & $p_{24}$ & $p_{31}$ & $p_{41}$ \\
    \midrule
    $0.0502$ & $0.0165$ & $0.0436$ & $0.1068$ & $0.04836$ \\
    \bottomrule
  \end{tabular}
  \caption{Parameter Estimates for Richter's 1976 Performance of Chopin's Mazurka Op. 63 No.3}
  \label{tab:parameterestimatesrichter}
\end{table}

\section{Conclusion}
\label{sec:conclusion}

\nocite{*}

\bibliographystyle{agsm}
\bibliography{musicdynamicsbib.bib}

\end{document}
