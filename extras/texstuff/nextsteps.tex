\documentclass[11pt]{article}

% Include statements
\usepackage{graphicx}
\usepackage{amsfonts,amssymb,amsmath,amsthm}
\usepackage[numbers,square,sort]{natbib}
\usepackage[left=1in,top=1in,right=1in,bottom=1in,nohead]{geometry}
\usepackage{pdfsync}
\usepackage{hyperref}
\hypersetup{backref,colorlinks=true,citecolor=blue,linkcolor=blue,urlcolor=blue}
\usepackage{algorithm,algorithmic}
\def\algorithmautorefname{Algorithm}


\usepackage[showonlyrefs]{mathtools}
\usepackage{pgf}
\usepackage{tikz}
\usetikzlibrary{fit,arrows,automata}					% fitting shapes to coordinates
\usetikzlibrary{backgrounds}
%\usepackage{autonum}


% Bibliography

\renewcommand*{\figureautorefname}{Figure}%
\renewcommand*{\tableautorefname}{Table}%
\renewcommand*{\partautorefname}{Part}%
\renewcommand*{\chapterautorefname}{Chapter}%
\renewcommand*{\sectionautorefname}{Section}%
\renewcommand*{\subsectionautorefname}{Section}%
\renewcommand*{\subsubsectionautorefname}{Section}% 


% Macros
\newcommand{\Year}[2]{\parbox[t]{2mm}{\multirow{#2}{*}{\rotatebox[origin=c]{90}{\bf
        #1}}}}
\DeclareMathOperator*{\argmin}{argmin}
\newcommand{\norm}[1]{\lVert #1 \rVert}
\newcommand{\R}{\mathbb{R}}
\newcommand{\E}{\mathbb{E}}
\newcommand{\X}{\mathbf{X}}
\newcommand{\Sig}{\boldsymbol{\Sigma}}
\newcommand{\Y}{\mathbf{Y}}
\newcommand{\one}{\mathbf{1}}
\newcommand{\Proj}{\boldsymbol{\Pi}}
\newcommand{\A}{\mathbf{A}}
%\renewcommand{\S}{\mathcal{S}}
\newcommand{\T}{\mathcal{T}}
\newcommand{\D}{\mathcal{D}}
\newcommand{\Q}{\mathbf{Q}}
\newcommand{\x}{\mathbf{x}}
\newcommand{\PP}{\mathcal{P}}
\newcommand{\Expect}[1]{\E\left[#1\right]}
\newcommand{\nik}[2]{\mathcal{N}_q(#1,#2)}
\newcommand{\nikol}{\nik{\beta}{C}}
\newcommand{\nikola}[1]{\mathcal{N}_#1(\beta,C)}
\newcommand{\given}{\mbox{ }\vert\mbox{ }}
\renewcommand{\hat}{\widehat}
\newcommand{\tr}{\textrm{tr}}
\renewcommand{\tilde}{\widetilde}
\newcommand{\ProjQ}{\Proj_{\mathbf{Q}}}
\newcommand{\TrueMeasure}{\nu}
\newcommand{\RademacherVariable}{\xi}
\newcommand{\RademacherComplexity}{\mathfrak{R}}
\newcommand{\LawOf}[1]{\mathcal{L}\left( #1 \right)}
\newcommand{\Expectwrt}[2]{\mathbb{E}_{ #1 }\left[ #2 \right]}
\newcommand{\ghost}{\widetilde{Z}}
\usepackage{xspace}
\makeatletter
\newcommand*{\iid}{%
    \@ifnextchar{.}%
        {i.i.d.}%
        {i.i.d.\@\xspace}%
}
\makeatother



\newtheorem{result}{Result}
\newtheorem{theorem}{Theorem}[result]
\renewcommand*{\sectionautorefname}{Section}%

\renewcommand{\P}{\mathbb{P}}
\DeclareMathOperator*{\argmax}{argmax}
\def\indep{\perp\!\!\!\perp}

\newcommand{\makeHeader}{\begin{center} 
DJM \hfill Ideas for moving forward \hfill 28 August 2017


\rule{\textwidth}{1pt}
\end{center}
}


\begin{document}
\suppressfloats
\makeHeader



\section{Background}

Musical recordings
are complex data files that describe the intensity and onset time
for every keystroke made by the performer. Matching this data to a
musical score, removing incorrect notes, anticipating note onsets for
automated accompaniment, comparing diverse performances, and
discovering the relationship between performer choice and listener
enjoyment all require ``smoothing'' the performance data so as to find
low-dimensional structure. Statistical
techniques like smoothing splines presume small changes in a
derivative. 
But musical performances do not conform to these assumptions because tempo and
dynamic interpretations rely on the juxtaposition of local smoothness
with sudden changes and emphases to create listener interest. It is
exactly the parts of a performance that are poorly described by
statistical smoothers that render a performance
interesting. Furthermore, many of these inflections are notated by the
composer or are implicit in performance practice developed over
centuries of musical expressivity. Consequently, regularization that
incorporates domain knowledge leads to better statistical and
empirical results~\citep{McDonald2016}. 

\begin{figure}
  \centering
  \includegraphics[width=0.75\textwidth]{hattoSplines.pdf}
  \caption{The tempo (beats/minute) of a 2003 recording attributed to Joyce
    Hatto.}
  \label{fig:music}
\end{figure}
\autoref{fig:music}
shows (blue dots) the note-by-note tempo of a 2003 recording
attributed to Joyce Hatto. Splines with
equally spaced knots (orange/dotted) are too smooth, and choosing locations
to duplicate knots manually (red/dashed)
to coincide with musical phrase endings works better. The solid green line 
shows a learned musical pattern from a Markov Switching
state-space model we developed which can
automatically learn tempo emphases (for example, near measure 40),
where the performer plays individual notes slightly slower than the
prevailing tempo, and automatically discover phrases
without purposeful knot 
duplication. Interestingly, such musical analyses can help to compare
performances---it was discovered in 2006 that this
particular recording was actually made in 1988 by Eugen
Indjic~\citep{CookSapp2009}. 

This application is especially fascinating since
it allows for visual, numerical, and aural exploration of the
effects of tuning parameter selection on inference. Working with
Prof.\ Chris Raphael on a
B\"{o}sendorfer CEUS reproducing piano in the IU Jacobs School
of Music, we can capture detailed 
measurements of key and pedal trajectories over time.  The
information is precise enough to reproduce an accurate replica by
artificially ``playing'' the piano just as was done during the
original performance. The piano can also create and respond to MIDI
data. However, statistical inferences for these data are difficult. Current
procedures are ill-suited for parameter estimation in a 
computationally efficient manner, as it amounts to Gaussian mixture
learning with $K^n$ components. Existing optimization
algorithms~\citep{Raphael2002,GhahramaniHinton2000} thus only
approximate the global solution. 


\section{The main idea}

\begin{itemize}
\item We want to model tempo and dynamic decisions.
\item We want a musician to understand what the parameters mean.
\end{itemize}


We will use a switching state-space model as shown in
\autoref{fig:switchss}. For now, we assume $s$ is a hidden Markov model on
four states, denoted $S_1,\ldots,S_4$ with transition probability
diagram given by \autoref{fig:transmat}.

\begin{figure}
\centering
% The continuous state vector is represented by a circle.
% "minimum size" makes sure all circles have the same size
% independently of their contents.
\tikzstyle{state}=[circle,
                                    thick,
                                    minimum size=1.2cm, draw=black]
% The (continuous) measurement vector is represented by an orange circle.
\tikzstyle{measurement}=[circle,
                                                thick,
                                                minimum size=1.2cm,
                                                draw=orange!80,
                                                fill=orange!25]
                                                \tikzstyle{switch}=[rectangle,
  thick, minimum size=1cm, draw=black]

\begin{tikzpicture}[>=latex,text height=1.5ex,text depth=0.25ex]
    % "text height" and "text depth" are required to vertically
    % align the labels with and without indices.
  
  % The various elements are conveniently placed using a matrix:
  \matrix[row sep=1cm,column sep=1cm] {
    % First line: Switch state
    \node (s_k-2)  {$\cdots$}; &
    \node (s_k-1) [switch]{$s_{k-1}$}; &
    \node (s_k)   [switch]{$s_k$};     &
    \node (s_k+1) [switch]{$s_{k+1}$}; &
    \node (s_k+2) {$\cdots$};
    \\
    % Second line: hidden continuous state
   \node (x_k-2) {$\cdots$}; &
   \node (x_k-1) [state] {$\mathbf{x}_{k-1}$}; &
   \node (x_k)   [state] {$\mathbf{x}_k$};     &
   \node (x_k+1) [state] {$\mathbf{x}_{k+1}$}; &
   \node (x_k+2) {$\cdots$};
   \\
   % Third line: Measurement
   \node (y_k-2) {$\cdots$}; &        
   \node (y_k-1) [measurement] {$\mathbf{y}_{k-1}$}; &
   \node (y_k)   [measurement] {$\mathbf{y}_k$};     &
   \node (y_k+1) [measurement] {$\mathbf{y}_{k+1}$}; &
   \node (y_k+2) {$\cdots$};
   \\
  };
    
    % The diagram elements are now connected through arrows:
    \path[->]
        (s_k-2) edge (s_k-1)
        (s_k-1) edge (s_k)	
        (s_k)   edge (s_k+1)	
        (s_k+1)   edge (s_k+2)	

      (x_k-2) edge (x_k-1)
      (x_k-1) edge (x_k)	
      (x_k)   edge (x_k+1)	
      (x_k+1)   edge (x_k+2)	
      
      (s_k-1) edge (x_k-1)
      (s_k) edge (x_k)
      (s_k+1)   edge (x_k+1)
      
      (x_k-1) edge (y_k-1)
      (x_k) edge (y_k)
      (x_k+1)   edge (y_k+1)
      
      (s_k-1) edge (x_k)
      (s_k) edge (x_k+1)
      (s_k+1)   edge (x_k+2)
      
      (s_k-1) edge[bend left] (y_k-1)
      (s_k) edge[bend left] (y_k)
      (s_k+1)   edge[bend left] (y_k+1)
        ;
\end{tikzpicture}

\caption{Switching state space model. Filled objects are observed,
  rectangles are discrete, and circles are continuous.\label{fig:switchss}}
\end{figure}


\begin{figure}[h!]
  \centering
  \tikzstyle{switch}=[rectangle,
  thick, minimum size=1cm, draw=black]
  \begin{tikzpicture}[>=latex,text height=1.5ex,text depth=0.25ex]
     \matrix[row sep=0.5cm,column sep=0.5cm] {
       \node (S1) [switch] {$S_1$}; & \node (S2) [switch] {$S_2$}; \\
       \node (S4) [switch] {$S_4$}; & \node (S3) [switch] {$S_3$}; \\
     };
     \path[->]
     (S1) edge [bend right] (S4)
     (S1) edge (S2)
     (S2) edge (S3)
     (S3) edge (S1)
     (S4) edge [bend right] (S1);
     \draw[->] (S1) to [out=90, in=180,looseness=4] (S1);
     \draw[->] (S2) to [out=90, in=0,looseness=4] (S2);
     \draw[->] (S3) to [out=270, in=0,looseness=4] (S3);
  \end{tikzpicture}
  \caption{Transition diagram. \label{fig:transmat}}
\end{figure}


Models like this can have many behaviors, but for our case, the
general form is:
\begin{align}
  x_{t}&= d(s_{t},s_{t-1})+T(s_{t},s_{t-1}) x_t + R(s_{t},s_{t-1})\eta_{t} & \eta_t &\sim
                                                      N(0,Q(s_{t},s_{t-1}))\\
  y_t&= c(s_t) + Z(s_t) x_t + \epsilon_t & \epsilon_t &\sim N(0, G(s_t)).
\end{align}
In other words, the hidden markov (switch) state determines which parameter
matrices govern the evolution of the system. 

The 4 switch states correspond to 4 different behaviors for the
performer: (1) constant tempo, (2) speeding up, (3) slowing down, and
(4) single note stress. The hidden continuous variable ($x_t$) is
taken to be a two component vector with the first component being the
``ideal'' tempo and the second being the acceleration. 
Corresponding to these configurations, the parameter
matrices are given in \autoref{tab:parmats}

\begin{table}[h!]
\centering
\begin{tabular}[h!]{cc|ccc}
  \hline\hline
  \multicolumn{5}{c}{Transition equation}\\
  \hline
  \multicolumn{2}{c|}{Switch states} & \multicolumn{3}{c}{Parameter
                                      matrices}\\
  $s_{t}$ & $s_{t-1}$ & $d$ & $T$ & $R$ \\
  \hline
  $S_1$ &  $S_1$ & 0 & $\begin{pmatrix}1&0\\0&0\end{pmatrix}$ & 0\\
  $S_2$ & $S_1$ & $\begin{pmatrix} l_t\tau_t\\ \tau_t\end{pmatrix}$ 
                                    & $\begin{pmatrix} 1 & 0 \\ 0 &
                                      0 \end{pmatrix}$ 
          & $\begin{pmatrix} 0 & l_t\\ 0 & 1 \end{pmatrix}$\\
  $S_4$ & $S_1$ & $\begin{pmatrix}0\\\varphi_t\end{pmatrix}$ 
                                     & $\begin{pmatrix}1&0\\0&0\end{pmatrix}$
          & $\begin{pmatrix}0&0\\0&1\end{pmatrix}$\\
  $S_2$ & $S_2$ & 0 & $\begin{pmatrix} 1 & l_t \\ 0 &
    1 \end{pmatrix}$ & 0\\
  $S_3$ & $S_2$ & $\begin{pmatrix} l_t\tau_t\\ \tau_t\end{pmatrix}$ 
                                    & $\begin{pmatrix} 1 & 0 \\ 0 &
                                      0 \end{pmatrix}$ 
          & $\begin{pmatrix} 0 & l_t\\ 0 & 1 \end{pmatrix}$\\
  $S_1$ & $S_3$ & $\begin{pmatrix} \mu_t\\0\end{pmatrix}$ & 0
          & $\begin{pmatrix} 1 & 0\\ 0 & 0 \end{pmatrix}$\\
  $S_3$ & $S_3$ & 0& $\begin{pmatrix} 1 & l_t \\ 0 &
    1 \end{pmatrix}$ & 0\\
  $S_1$ &  $S_4$ & 0 & $\begin{pmatrix}1&0\\0&0\end{pmatrix}$ & 0\\
  \hline\hline
  \multicolumn{5}{c}{Measurement equation}\\
  \hline
  \multicolumn{2}{c|}{Switch states} & \multicolumn{3}{c}{Parameter
                                      matrices}\\
  $s_t$ && $c$ & $Z$ & $G$\\
  \hline
  $S_4$ & & 0 & $\begin{pmatrix} 1 & 1 \end{pmatrix}$ &
                                                                  $\sigma^2_\epsilon$\\
  else && 0 & $\begin{pmatrix} 1 & 0 \end{pmatrix}$ &
                                                                  $\sigma^2_\epsilon$\\
  \hline\hline
\end{tabular}
\caption{Parameter matrices of the switching state space model.\label{tab:parmats}}
\end{table}
Finally, 
\[
Q=\begin{cases}\begin{pmatrix} \sigma^2_1 &
    0\\0&\sigma^2_4\end{pmatrix} & (s_t,\ s_{t-1})=(S_4,\ S_1)\\
\begin{pmatrix} \sigma^2_1 &
    0\\0&\sigma^2_2\end{pmatrix} & \textrm{else}.
\end{cases}
\] 
So for any performance, we want to be able to estimate
the following parameters: $\sigma_1^2$, $\sigma_2^2$, $\sigma^2_4$,
$\sigma_\epsilon^2$, the probabilities of the transition matrix (there
are 4), and vectors $\mu$, $\tau$, and $\varphi$. These last three
will be of different lengths depending on the number of times the
state is visited. Lastly, we have the initial state distributions
\[
x_1\sim\begin{cases} N\left( \begin{pmatrix}\mu_1\\0\end{pmatrix}
  ,\ \begin{pmatrix} \sigma^2_1 & 0\\0 & 0
  \end{pmatrix}\right) & s_1=S_1\\
N\left( \begin{pmatrix}\mu_1\\\tau_1\end{pmatrix}
  ,\ \begin{pmatrix} \sigma^2_1 & 0\\0 & \sigma^2_2
  \end{pmatrix}\right) & s_1=S_3.\end{cases}
\]
Importantly, this is just one way to write this
model.

The \texttt{R} function
\texttt{yupengMats} creates all of these different parameter matrices.


\section{R documentation}

The problem with estimating a model like this is that, because the
switch states and the continuous states are both hidden, this becomes
an NP-hard problem. In particular, there are $4^N$ possible paths
through the switch variables, so evaluating the likelihood at all of
them is intractable. Thus, I implemented a particular
approximation. The Beam Search (\autoref{alg:beamsearch}), finds
(greedily) evaluates the most likely path through the switch
states. Another name for the algorithm is Discrete Particle Filter
(\texttt{dpf}). Once we have those, \texttt{getLogLike} returns the
negative 
loglikelihood of the data associated with that path. So for any
configuration of parameters, we would form the matrices
(\texttt{yupengMats}) then find the best path (\texttt{dpf}) then
evaluate the likelihood of that path (\texttt{getLogLike}). We can
then optimize over parameters using any variety of numerical
optimization technique. However, when I have tried this, I always get
infinite likelihood.

For this model, the dpf is more easily specified if we make the
measurement equation depend only on the current state and not the
previous state. For this reason, the code uses 16 states rather than
4. One can always change a Markov model in this way.

\begin{algorithm}[t!]
  \caption{Beam search\label{alg:beamsearch}}
  \begin{algorithmic}[1]
  \STATE {\bfseries Input:}
  Initial parameters of the matrices. Integer beam width $B$.
  \FOR{$i=1$ {\bfseries to} $N$} 
  \STATE (\texttt{dpf} performs 1-step of the following);
  \STATE For each current path, calculate the 1-step likelihood for
  moving to each potential switch (\texttt{kf1step})\;
  \STATE Multiply the likelihood by the probability of transitioning
  to that switch state\;
  \STATE Multiply by the previous path weights $w$\;
  \STATE If $\norm{w}_0>B$, resample the weights\;
  (\texttt{resampleSubOptimal}) to get $B$ non-zero weights which
  add to 1.\;
  \STATE Keep only those paths corresponding to the non-zero weights\;
  \ENDFOR
  \STATE Return $B$ paths through the switch space along with their weights.\;
\end{algorithmic}
\end{algorithm}


\section{Python}

A friend of mine wrote a python library for switching state space
models (\href{https://github.com/mattjj/pyslds}{mattjj/pyslds}). It
may be easier to use this rather than the code I wrote. It is Bayesian
rather than frequentist. If you're familiar with python, feel free to
give it a whirl.

\bibliographystyle{mybibsty}
\bibliography{AllReferences}
\end{document}

